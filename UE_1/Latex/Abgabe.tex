% Created 2020-10-25 So 23:03
% Intended LaTeX compiler: pdflatex
\documentclass[11pt]{scrartcl}
\usepackage[utf8]{inputenc}
\usepackage[T1]{fontenc}
\usepackage{graphicx}
\usepackage{grffile}
\usepackage{longtable}
\usepackage{wrapfig}
\usepackage{rotating}
\usepackage[normalem]{ulem}
\usepackage{amsmath}
\usepackage{textcomp}
\usepackage{amssymb}
\usepackage{capt-of}
\usepackage{MnSymbol}
\usepackage{mathtools}
\usepackage{setspace}
\usepackage[straightvoltages, european resistors, american inductors]{circuitikz}
\author{David Keller, Moritz Woltron, Matthias Fottner}
\date{}
\title{Protokoll Übung 1}
\begin{document}

\maketitle

\section{Schaltplan mit allen Strömen und Spannungen}
\begin{circuitikz}[scale=1.2]
        \draw (0,0) node(qqqn0){};
        \draw (0,0) to[isource=$I_{S1}$, o-o] (0,4) node[label=left:$K_2$](n2){};
        \draw (n2) to[R=$R_1$, i=$I_{R1}$, v<=$U_{R1}$] ++ (0,4) to[short] ++ (4,0) node[label=above:$K_1$](n1){};
        \draw (n1) to[vsource=$U_{S2}$, i>=$I_{S2}^{?}$,  o-o] ++ (0, -4) node[label=north east:$K_3$](n3){};
        \draw (n2) to[R=$R_2$, i=$I_{R2}$, v=$U_{R2}$] (n3);
        \draw (n3) to[R=$R_5$, i=$I_{R5}$, v=$U_{R5}$, o-o] ++ (4,0) node[label=right:$K_4$](n4){};
        \draw (n1) to[short] ++ (4,0) to[cvsource=$U_{S3}$, i>=$I_{S3}^{?}$] (n4);
        \draw (n3) to[short] ++ (0,-1.5) node(n30){};
        \draw (n30) to[R=$R_3$, i=$I_{R3}$, v=$U_{R3}$] (4,0);
        \draw (n30) to[short] ++ (2.5,0) to[R=$R_4$, i=$I_{R4}$, v=$U_{R4}$] (6.5,0);
        \draw (6.5,0) to[short] (0,0);
        \draw (0,0) node[rground]{};
\end{circuitikz}

\newpage
\section{Kirchhoff'schen Knotengleichungen}

\begin{doublespace}
K1: \(\displaystyle \quad I_{R1} + I_{S2}^? + I_{S3}^? = 0\) \\
K2: \(\displaystyle \quad I_{R2} - I_{S1} - I_{R1} = 0\)\\
K1: \(\displaystyle \quad I_{R5} + I_{R3} + I_{R4} - I_{R2} - I_{S2}^? = 0\) \\
K1: \(\displaystyle \quad -I_{R5} - I_{S3}^? = 0\)
\end{doublespace}

\subsection{Ohm'sches Gesetz}
\begin{spacing}{2.5}
K1: \(\displaystyle \quad \frac{U_{R1}}{R_1} + I_{S2}^? + I_{S3}^? = 0\) \\
K2: \(\displaystyle \quad \frac{U_{R2}}{R_2} - \frac{U_{R1}}{R_1} = I_{S1}\) \\
K3: \(\displaystyle \quad \frac{U_{R5}}{R_5} + \frac{U_{R3}}{R_3} + \frac{U_{R4}}{R_4} - \frac{U_{R2}}{R_2} - I_{S2}^? = 0\) \\
K4: \(\displaystyle \quad - \frac{U_{R5}}{R_5} - I_{S3}^? = 0\)
\end{spacing}

\subsection{Knotenspannungen}
\begin{doublespace}
\(\displaystyle U_{R3} = U_{R4} = U_{n3}\) \\
\(\displaystyle U_{R2} = U_{n2}- U_{n3}\) \\
\(\displaystyle U_{R1} = U_{n1}- U_{n2}\) \\
\(\displaystyle U_{R5} = U_{n3}- U_{n4}\)
\end{doublespace}

\subsection{Knotengleichungen mit Knotenspannungen}
\begin{spacing}{2.5}
  K1: \(\displaystyle \quad \frac{U_{n1} - U_{n2}}{R_1} + I_{S2}^? + I_{S3}^? = 0\) \\
  K2: \(\displaystyle \quad \frac{U_{n2} - U_{n3}}{R_2} - \frac{U_{n1} - U_{n2}}{R_1} = I_{S1}\) \\
  K3: \(\displaystyle \quad \frac{U_{n3} - U_{n4}}{R_5} + \frac{U_{n3}}{R_3} + \frac{U_{n3}}{R_4} - \frac{U_{n2} - U_{n3}}{R_2} - I_{S2}^?= 0\) \\
  K4: \(\displaystyle \quad - \frac{U_{n3} - U_{n4}}{R_5} - I_{S3}^? = 0\)
\end{spacing}
\subsection{zusätzliche Bedingungen}
\begin{itemize}
\item 6 Unbekannte: $U_{n1}, U_{n2}, U_{n3}, U_{n4}, I_{S2}^?, I_{S3}^?$
\item nur 4 Gleichungen: unbestimmtes Gleichungssystem
\item 2 weitere Gleichungen
\end{itemize}
\begin{align}
  U_{S2} &= U_{n1} - U_{n3} \\
  U_{S3} &= U_{n1} - U_{n4} = \alpha \cdot I_{R3} = \alpha \frac{U_{R3}}{R_3} = \alpha \frac{U_{n3}}{R_3} \nonumber \\
  \Longrightarrow 0 &= U_{n1} - U_{n4} - \alpha \frac{U_{n3}}{R_3}
\end{align}


\section{Gleichungssystem in Matrixform}




\end{document}
