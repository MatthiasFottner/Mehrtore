% Created 2020-10-25 So 23:03
% Intended LaTeX compiler: pdflatex
\documentclass[11pt]{scrartcl}
\usepackage[utf8]{inputenc}
\usepackage[T1]{fontenc}
\usepackage{graphicx}
\usepackage{grffile}
\usepackage{longtable}
\usepackage{wrapfig}
\usepackage{rotating}
\usepackage[normalem]{ulem}
\usepackage{amsmath}
\usepackage{textcomp}
\usepackage{amssymb}
\usepackage{capt-of}
\usepackage{MnSymbol}
\usepackage{mathtools}
\usepackage{setspace}
\usepackage{nicematrix}
\usepackage[straightvoltages, european resistors, american inductors]{circuitikz}
\author{David Keller, Moritz Woltron, Matthias Fottner}
\date{}
\title{Protokoll Übung 1}
\begin{document}

\maketitle

\section{Schaltplan mit allen Strömen und Spannungen}
\begin{circuitikz}[scale=1.2]
        \draw (0,0) node(qqqn0){};
        \draw (0,0) to[isource=$I_{S1}$, o-o] (0,4) node[label=left:$K_2$](n2){};
        \draw (n2) to[R=$R_1$, i=$I_{R1}$, v<=$U_{R1}$] ++ (0,4) to[short] ++ (4,0) node[label=above:$K_1$](n1){};
        \draw (n1) to[vsource=$U_{S2}$, i>=$I_{S2}^{?}$,  o-o] ++ (0, -4) node[label=north east:$K_3$](n3){};
        \draw (n2) to[R=$R_2$, i=$I_{R2}$, v=$U_{R2}$] (n3);
        \draw (n3) to[R=$R_5$, i=$I_{R5}$, v=$U_{R5}$, o-o] ++ (4,0) node[label=right:$K_4$](n4){};
        \draw (n1) to[short] ++ (4,0) to[cvsource=$U_{S3}$, i>=$I_{S3}^{?}$] (n4);
        \draw (n3) to[short] ++ (0,-1.5) node(n30){};
        \draw (n30) to[R=$R_3$, i=$I_{R3}$, v=$U_{R3}$] (4,0);
        \draw (n30) to[short] ++ (2.5,0) to[R=$R_4$, i=$I_{R4}$, v=$U_{R4}$] (6.5,0);
        \draw (6.5,0) to[short] (0,0);
        \draw (0,0) node[rground]{};
\end{circuitikz}

\newpage
\section{Kirchhoff'schen Knotengleichungen}

\begin{doublespace}
K1: \(\displaystyle \quad I_{R1} + I_{S2}^? + I_{S3}^? = 0\) \\
K2: \(\displaystyle \quad I_{R2} - I_{S1} - I_{R1} = 0\)\\
K1: \(\displaystyle \quad I_{R5} + I_{R3} + I_{R4} - I_{R2} - I_{S2}^? = 0\) \\
K1: \(\displaystyle \quad -I_{R5} - I_{S3}^? = 0\)
\end{doublespace}

\subsection{Ohm'sches Gesetz}
\begin{spacing}{2.5}
K1: \(\displaystyle \quad \frac{U_{R1}}{R_1} + I_{S2}^? + I_{S3}^? = 0\) \\
K2: \(\displaystyle \quad \frac{U_{R2}}{R_2} - \frac{U_{R1}}{R_1} = I_{S1}\) \\
K3: \(\displaystyle \quad \frac{U_{R5}}{R_5} + \frac{U_{R3}}{R_3} + \frac{U_{R4}}{R_4} - \frac{U_{R2}}{R_2} - I_{S2}^? = 0\) \\
K4: \(\displaystyle \quad - \frac{U_{R5}}{R_5} - I_{S3}^? = 0\)
\end{spacing}

\subsection{Knotenspannungen}\label{sec:knotenspannungen}
\begin{doublespace}
\(\displaystyle U_{R3} = U_{R4} = U_{n3}\) \\
\(\displaystyle U_{R2} = U_{n2}- U_{n3}\) \\
\(\displaystyle U_{R1} = U_{n1}- U_{n2}\) \\
\(\displaystyle U_{R5} = U_{n3}- U_{n4}\)
\end{doublespace}

\subsection{Knotengleichungen mit Knotenspannungen}
\begin{spacing}{2.5}
  K1: \(\displaystyle \quad \frac{U_{n1} - U_{n2}}{R_1} + I_{S2}^? + I_{S3}^? = 0\) \\
  K2: \(\displaystyle \quad \frac{U_{n2} - U_{n3}}{R_2} - \frac{U_{n1} - U_{n2}}{R_1} = I_{S1}\) \\
  K3: \(\displaystyle \quad \frac{U_{n3} - U_{n4}}{R_5} + \frac{U_{n3}}{R_3} + \frac{U_{n3}}{R_4} - \frac{U_{n2} - U_{n3}}{R_2} - I_{S2}^?= 0\) \\
  K4: \(\displaystyle \quad - \frac{U_{n3} - U_{n4}}{R_5} - I_{S3}^? = 0\)
\end{spacing}

\subsection{zusätzliche Bedingungen}
\begin{itemize}
\item 6 Unbekannte: $U_{n1}, U_{n2}, U_{n3}, U_{n4}, I_{S2}^?, I_{S3}^?$
\item nur 4 Gleichungen: unbestimmtes Gleichungssystem
\item 2 weitere Gleichungen
\end{itemize}
\begin{align}
  U_{S2} &= U_{n1} - U_{n3} \\
  U_{S3} &= U_{n1} - U_{n4} = \alpha \cdot I_{R3} = \alpha \cdot \frac{U_{R3}}{R_3} = \alpha \cdot \frac{U_{n3}}{R_3} \nonumber \\
  \Longrightarrow 0 &= U_{n1} - U_{n4} - \alpha \cdot \frac{U_{n3}}{R_3}
\end{align}


\section{Gleichungssystem in Matrixform}
Definition Leitwerte: \(\displaystyle \quad G_n := \frac{1}{R_n}\)


\begin{equation*}
  \renewcommand{\arraystretch}{2}
    \underbrace{\left[\begin{NiceArray}{c:c:c:c:c:c}
    G_1 & -G_1 & 0 & 0 & 1 & 1 \\
    \hdottedline
    -G_1 & G_1 - G_2 & -G_2 & 0 & 0 & 0 \\
    \hdottedline
    0 & -G_2 & G_2 + G_3 + G_4 + G_5 & -G_5 & -1 & 0 \\
    \hdottedline
    0 & 0 & -G_5 & G_5 & 0 & -1 \\
    \hdottedline
    1 & 0 & -1 & 0 & 0 & 0 \\
    \hdottedline
    1 & 0 & -\alpha \cdot G_3 & -1 & 0 & 0
  \end{NiceArray}\right]}_{\text{A}} \ \cdot \
\underbrace{\left\{ \begin{NiceArray}{c}
    U_{n1} \\
    \hdottedline
    U_{n2} \\
    \hdottedline
    U_{n3} \\
    \hdottedline
    U_{n4} \\
    \hdottedline
    I_{S2}^? \\
    \hdottedline
    I_{S3}^?
  \end{NiceArray}\right\}}_{\text{x}} \ = \
\underbrace{\left\{ \begin{NiceArray}{c}
    0 \\
    \hdottedline
    I_{S1} \\
    \hdottedline
    0 \\
    \hdottedline
    0 \\
    \hdottedline
    U_{S2} \\
    \hdottedline
    0
  \end{NiceArray}
\right\}}_{\text{b}}
\end{equation*} \\

Nach x auflösen: \(\displaystyle \quad x = A^{-1} b \) \\

Mithilfe von Matlab erhält man:
\begin{equation*}
  \renewcommand{\arraystretch}{1.25}
  x = \left\{\begin{array}{c}
      U_{n1} \\
      U_{n2} \\
      U_{n3} \\
      U_{n4} \\
      I_{S2}^? \\
      I_{S3}^? \end{array}\right\} =
  \left\{ \begin{array}{c}
            5,2143 V\\
    3,5476 V\\
    1,7143 V\\
    3,5000 V\\
    -1,1310 A\\
    0,2976 A \end{array}\right\}
\end{equation*}

Mit diesen Werten und mithilfe der Knotenspannungsgleichungen aus \ref{sec:knotenspannungen} lassen sich die
Spannungen $U_{R1} - U_{R5}$ und die Ströme $I_{R1} - I_{R5}$ ermitteln:

\begin{equation*}
  \renewcommand{\arraystretch}{1.25}
  U = \left\{ \begin{array}{c}
                U_{R1} \\
                U_{R2} \\
                U_{R3} \\
                U_{R4} \\
                U_{R5} \\
                \end{array}
  \right\} =
  \left\{ \begin{array}{c}
                U_{n1} - U_{n2}\\
                U_{n2} - U_{n3} \\
                U_{n3} \\
                U_{n3} \\
                U_{n3} - U_{n4}
          \end{array}\right\} =
        \left\{ \begin{array}{c}
                  1,6667 V \\
                  1,8333 V \\
                  1,7143 V \\
                  1,7143 V \\
                  -1.7857 V
                \end{array}\right\}
            \end{equation*}

            \begin{equation*}
              \renewcommand{\arraystretch}{1.25}
              I = \left\{ \begin{array}{c}
                            I_{R1} \\
                            I_{R2} \\
                            I_{R3} \\
                            I_{R4} \\
                            I_{R5}
                          \end{array}\right\} =
                        \left\{ \begin{array}{c}
                                  U_{R1} / R_1 \\
                                  U_{R2} / R_2 \\
                                  U_{R3} / R_3 \\
                                  U_{R4} / R_4 \\
                                  U_{R5} / R_5
                                \end{array}\right\} =
                              \left\{ \begin{array}{c}
                                        0.8333 A \\
                                        1.8333 A \\
                                        0.4286 A \\
                                        0.5714 A \\
                                        -0.2976 A
                                      \end{array}\right\}
                                  \end{equation*} 

                                  Die Quellspannung der stromgesteuerten Quelle $U_{S3}$ erhält man aus der Lösung der Knotenspannungen:
                                  \begin{equation*}
                                    U_{S3} = U_{n1} - U_{n4} = 1,7143 V
                                  \end{equation*}
\end{document}
