% Created 2020-10-25 So 23:03
% Intended LaTeX compiler: pdflatex
\documentclass[11pt]{scrartcl}
\usepackage[utf8]{inputenc}
\usepackage[T1]{fontenc}
\usepackage[ngerman]{babel}
\usepackage{graphicx}
\usepackage{grffile}
\usepackage{longtable}
\usepackage{wrapfig}
\usepackage{rotating}
\usepackage[normalem]{ulem}
\usepackage{amsmath}
\usepackage{textcomp}
\usepackage{amssymb}
\usepackage{capt-of}
\usepackage{MnSymbol}
\usepackage{mathtools}
\usepackage{setspace}
\usepackage{nicematrix}
\usepackage{varwidth}
% \usepackage{getlab}
\usepackage{listings}
\usepackage{pdfpages}
\usepackage[straightvoltages, european resistors, american inductors]{circuitikz}
\usetikzlibrary{arrows.meta}
\tikzset{FARROW/.style={arrows={-Triangle[angle=45:2.0mm]}}}


\definecolor{darkspringgreen}{rgb}{0.09, 0.45, 0.27}    % Farbe für die Kommentare bei Listings
\lstset{
  language= Matlab,                     % Setzt die Sprache
  basicstyle=\scriptsize\ttfamily,     % Setzt den Standardstil
  % keywordstyle=\color{red}\bfseries,    % Setzt den Stil für Schlüsselwörter
  identifierstyle=\color{blue},        % Identifier bekommen keine gesonderte formatierung
  commentstyle=\color{darkspringgreen},        % Stil für Kommentare
  stringstyle=\ttfamily,             % Stil für Strings (gekennzeichnet mit "String")
  breaklines=true,             % Zeilen werden umgebrochen
  numbers=left,                 % Zeilennummern links
  numberstyle=\tiny,             % Stil für die Seitennummern
  frame=single,                 % Rahmen
  % backgroundcolor=\color{myGrey},     % Hintergrundfarbe
  % caption={Java-Code},             % Caption
  tabsize=2                % Größe der Tabulatoren
}



\newcommand{\GETHeader}[6]{%
  \begin{titlepage}
    \pagestyle{empty} \enlargethispage*{25cm}\samepage{
      \vspace*{-2.5cm}
      \begin{center}
        %
        \begin{tabular}[b]{lr}
          %
          \hspace*{-1.5cm}
          \begin{tabular}{p{3.8cm}}
            \includegraphics[width=3.8cm]{./Figures/TUGraz}
          \end{tabular}
          %
          \hspace*{-1cm}
          %
          %
          \begin{tabular}{p{13.8cm}}
            \begin{flushright}
              \large
              Institut für Grundlagen und Theorie der Elektrotechnik\\
              ~\\
            \end{flushright}
          \end{tabular}
        \end{tabular}

        \vspace*{2.2cm}
        %
        \Huge {Elektrische Netzwerke und Mehrtore \\ Übung\\} %
        \vspace*{.5cm} \Large{ Wintersemester 2020\\}
        %
        \vspace*{1.5cm}

        \Huge{\textbf{#1}\\}

        %
        % \vspace*{0.8cm} \Large{Übungsdatum: {#2}\\}
        %
        \vspace*{1cm} \vfill
        %
        \Large{Gruppe: {#2}\\} \vspace*{0.5cm}%


        % \Large{Protokollführer(in): {#3}\\} \vspace*{1cm}

        \Large{Gruppenteilnehmer:\\} \vspace*{.1cm}
        % Name der beteiligten Studierenden
        \Large{#3} \vspace{1cm}
        %
        %
        \Large{Vortragende: #4\\} \vspace*{.1cm}   %\vspace*{1.5cm}
        % \Large{Betreuer(in): #6\\}
        \vspace*{1.5cm}
        %
        \Large{#5, am #6}
      \end{center}}%
    %
    \clearpage
  \end{titlepage}}





\begin{document}

\GETHeader                                                                              %  Bitte Ausfüllen!!!
% ----------------------------
{Protokoll Übung 3: \\ Schaltvorgang Kondensator}                         %  Übungstitel
% ----------------------------
% {25. Mai 2020}                                                                  %  Übungsdatum
% ----------------------------
{04}                                                                   %  Gruppen-Nr.
% ----------------------------
% {Matthias Fottner}                                                                      % Name des Protokollführers oder der Protokollführerin
% ----------------------------
{
  \begin{center}
    \begin{varwidth}{\textwidth}
    \begin{enumerate}
    \item Matthias Fottner
    \item David Keller
    \item  Moritz Woltron
    \end{enumerate}
    \end{varwidth}
\end{center}
}
% ----------------------------
{Helena Grabner}                                                                     %  Laborleiter(in)
% {Übung 2}                                                               %  Betreuer(in)
% ----------------------------
{Graz}                                                                                  %  Ort der Protokollerstellung
{\today}                                                                %  Datum Protokollerstellung


\newcommand{\unit}[1]{\,\text{#1}}


\tableofcontents

\newpage



\section{Bestimmen des Anfangszustands von $u_C$} % David
\subsection{Schaltplan zur Schalterposition a}

\begin{figure}[!htb]
\begin{center}
	\begin{circuitikz} [european resistors, scale=1]
		\clip (-2,-5.7) rectangle (12.5, 5);

		\draw(0,0) to [R=$R_3$, v={\color{blue}{$U_{R3}$}}, i={\color{red}{$I_{R3}$}}, *-*] (4,0);
		\draw(4,0) to [R=$R_4$, v={\color{blue}{$U_{R4}$}}, i={\color{red}{$I_{R4}$}}] (4,4);
		\draw(4,0) to [R=$R_2$, v<={\color{blue}{$U_{R2}$}}, i<={\color{red}{$I_{R2}$}}] (4,-4);
		
		
		
		\draw(0,0) node[label={[font=\footnotesize]-190:n3}] {} to (0,4);
		\draw(0,4) to[isource, i<={\color{red}{$I_{S2}$}}, -*] (4,4) node[label={[font=\footnotesize]90:n2}] {};
		\draw(4,4) to [R=$R_5$, v={\color{blue}{$U_{R5} = 0\unit{A}$}}, i={\color{red}{$I_{R5} = 0\unit{A}$}}, -o] (8,4);
		\draw(8,4) to (10,4);
		
		\draw(0,0) to[vsource, v<={\color{blue}{$U_{S1}$}}, i<={\color{red}{$I_{S1}^?$}}] (0,-4);
		\draw(0,-4) to [R=$R_1$, v={\color{blue}{$U_{R1}$}}, i={\color{red}{$I_{R1}$}}, *-*] (4,-4) node[rground]{};
		\draw(4,-4) to [short, -o] (8,-4);
		\draw(8,-4) to (10,-4);
		\draw(10,-4) to [C, v<={\color{blue}{$u_{C} = 0\unit{A}$}}, i<={\color{red}{$i_{C} = 0\unit{A}$}}] (10,4);
		
		\draw[FARROW, green!50!black] (-0.2, 0) to[bend left=-90, looseness=2.2] node[below left]{$U_{n3}$} (3.9,-4.2);
		\draw[FARROW, green!50!black] (4.1, 3.95) to[bend right=-75, looseness=0.9] node[below right]{$U_{n4}$} (4.4,-3.9);
		\draw[FARROW, green!50!black] (4.1, 0) to[bend right=-75, looseness=0.7] node[below right]{$U_{n1}$} (4.4,-3.7);
		\draw[FARROW, green!50!black] (0.1, -3.8) to[bend right=-75, looseness=0.5] node[above]{$U_{n2}$} (3.9,-3.7);
		
		%\draw [brown] (current bounding box.south west) rectangle (current bounding box.north east);
		%\draw[gray,step=0.25] (-6,-3) grid (16.5, 11);
		
		
	\end{circuitikz}
\end{center}
\caption{Netzwerk mit allen eingezeichneten Strömen, (Knoten-)spannungen und Knoten}
\label{fig:schaltplan_a}
\end{figure}

\pagebreak

\subsection{Erstellen der erweiterten KSV-Matrix}

Um die Matrix des erweiterten KSVs aufstellen zu können, muss zu den 4 Knotengleichungen der unbekannte Strom der Spannungsquelle $U_{S1}$ in Form einer 5. Gleichung hinzugefügt werden.

\begin{equation*}
{\color{blue}{U_{S1}}} = U_{n2}-U_{n3}
\end{equation*}

\begin{align*}
\renewcommand{\arraystretch}{1.5}
\begin{bNiceArray}[first-row, first-col]{c:c:c:c:c}
& n1 & n2 & n3 & n4 & \color{red}{I_{S1}^?} \\
n1 & G_2+G_3+G_4 & 0 & -G_3 & -G_4 & 0 \\
\hdottedline
n2 & 0 & G_1 & 0 & 0 & 1 \\
\hdottedline
n3 & -G_3 & 0 & G_3 & 0 & -1 \\
\hdottedline
n4 & -G_4 & 0 & 0 & G_4 & 0 \\
\hdottedline
\color{red}{I_{S1}^?} & 0 & 1 & -1 & 0 & 0
\end{bNiceArray}
\begin{Bmatrix}
U_{n1} \\
U_{n2} \\
U_{n3} \\
U_{n4} \\
\color{red}{I_{S1}^?}
\end{Bmatrix} =
\begin{Bmatrix}
0 \\
0 \\
\color{red}{I_{S2}} \\
\color{red}{-I_{S2}}\\
\color{blue}{U_{S1}}
\end{Bmatrix}\\
\end{align*}

Man erhält mithilfe von Matlab für $x$:
\begin{equation*}
\renewcommand{\arraystretch}{1.5}
x = \begin{Bmatrix}
U_{n1} \\
U_{n2} \\
U_{n3} \\
U_{n4} \\
I_{S1}^?
\end{Bmatrix} =
\begin{Bmatrix}
-3,36 \unit{V} \\
2,24 \unit{V} \\
-7,76 \unit{V} \\
-4,32 \unit{V} \\
-0,56 \unit{A}
\end{Bmatrix}
\end{equation*}

\subsection{Bestimmen von $u_C$}

Wie sich im Schaltplan in Abbildung \ref{fig:schaltplan_a} erkennen lässt, entspricht $U_{C,a} = U_{n4}$:

\begin{equation*}
U_{C,a} = U_{n4} = -4,32 \unit{V}
\end{equation*}

\section{Aufstellen der Differentialgleichung} % Matthias
\subsection{Schaltplan zur Schalterposition b}
\subsection{Erstellen der KSV-Matrix}
\subsection{Lösen der Differentialgleichung}
\subsubsection{Homogene Lösung}
\subsubsection{Inhomogene Lösung}
\subsubsection{Anfangswertproblem}
\subsubsection{Gesamtlösung}
\section{Vergleich mit allgemeiner Lösungsformel}
\section{Simulation in PSpice}
\section{Matlab-Skript}


\end{document}