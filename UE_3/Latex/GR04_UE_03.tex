% Created 2020-10-25 So 23:03
% Intended LaTeX compiler: pdflatex
\documentclass[11pt]{scrartcl}
\usepackage[utf8]{inputenc}
\usepackage[T1]{fontenc}
\usepackage[ngerman]{babel}
\usepackage{graphicx}
\usepackage{grffile}
\usepackage{longtable}
\usepackage{wrapfig}
\usepackage{rotating}
\usepackage[normalem]{ulem}
\usepackage{amsmath}
\usepackage{trfsigns}
\usepackage{textcomp}
\usepackage{amssymb}
\usepackage{capt-of}
\usepackage{MnSymbol}
\usepackage{mathtools}
\usepackage{setspace}
\usepackage{nicematrix}
\usepackage{varwidth}
% \usepackage{getlab}
\usepackage{listings}
\usepackage{pdfpages}
\usepackage[straightvoltages, european resistors, american inductors]{circuitikz}
\usetikzlibrary{arrows.meta}
\tikzset{FARROW/.style={arrows={-Triangle[angle=45:2.0mm]}}}


\definecolor{darkspringgreen}{rgb}{0.09, 0.45, 0.27}    % Farbe für die Kommentare bei Listings
\lstset{
  language= Matlab,                     % Setzt die Sprache
  basicstyle=\scriptsize\ttfamily,     % Setzt den Standardstil
  % keywordstyle=\color{red}\bfseries,    % Setzt den Stil für Schlüsselwörter
  identifierstyle=\color{blue},        % Identifier bekommen keine gesonderte formatierung
  commentstyle=\color{darkspringgreen},        % Stil für Kommentare
  stringstyle=\ttfamily,             % Stil für Strings (gekennzeichnet mit "String")
  breaklines=true,             % Zeilen werden umgebrochen
  numbers=left,                 % Zeilennummern links
  numberstyle=\tiny,             % Stil für die Seitennummern
  frame=single,                 % Rahmen
  % backgroundcolor=\color{myGrey},     % Hintergrundfarbe
  % caption={Java-Code},             % Caption
  tabsize=2                % Größe der Tabulatoren
}



\newcommand{\GETHeader}[6]{%
  \begin{titlepage}
    \pagestyle{empty} \enlargethispage*{25cm}\samepage{
      \vspace*{-2.5cm}
      \begin{center}
        %
        \begin{tabular}[b]{lr}
          %
          \hspace*{-1.5cm}
          \begin{tabular}{p{3.8cm}}
            \includegraphics[width=3.8cm]{./Figures/TUGraz}
          \end{tabular}
          %
          \hspace*{-1cm}
          %
          %
          \begin{tabular}{p{13.8cm}}
            \begin{flushright}
              \large
              Institut für Grundlagen und Theorie der Elektrotechnik\\
              ~\\
            \end{flushright}
          \end{tabular}
        \end{tabular}

        \vspace*{2.2cm}
        %
        \Huge {Elektrische Netzwerke und Mehrtore \\ Übung\\} %
        \vspace*{.5cm} \Large{ Wintersemester 2020\\}
        %
        \vspace*{1.5cm}

        \Huge{\textbf{#1}\\}

        %
        % \vspace*{0.8cm} \Large{Übungsdatum: {#2}\\}
        %
        \vspace*{1cm} \vfill
        %
        \Large{Gruppe: {#2}\\} \vspace*{0.5cm}%


        % \Large{Protokollführer(in): {#3}\\} \vspace*{1cm}

        \Large{Gruppenteilnehmer:\\} \vspace*{.1cm}
        % Name der beteiligten Studierenden
        \Large{#3} \vspace{1cm}
        %
        %
        \Large{Vortragende: #4\\} \vspace*{.1cm}   %\vspace*{1.5cm}
        % \Large{Betreuer(in): #6\\}
        \vspace*{1.5cm}
        %
        \Large{#5, am #6}
      \end{center}}%
    %
    \clearpage
  \end{titlepage}}




\newcommand{\vlaplace}[1][]{\mbox{\setlength{\unitlength}{0.1em}%
    \begin{picture}(10,20)%
      \put(3,2){\circle{4}}%
      \put(3,4){\line(0,1){12}}%
      \put(3,18){\circle*{4}}%
      \put(10,7){#1}
    \end{picture}%
  }%
}%

\newcommand{\vLaplace}[1][]{\mbox{\setlength{\unitlength}{0.1em}%
    \begin{picture}(10,20)%
      \put(3,2){\circle*{4}}%
      \put(3,4){\line(0,1){12}}%
      \put(3,18){\circle{4}}%
      \put(10,7){#1}
    \end{picture}%
  }%
}%














\begin{document}

\GETHeader                                                                              %  Bitte Ausfüllen!!!
% ----------------------------
{Protokoll Übung 3: \\ Schaltvorgang Kondensator}                         %  Übungstitel
% ----------------------------
% {25. Mai 2020}                                                                  %  Übungsdatum
% ----------------------------
{04}                                                                   %  Gruppen-Nr.
% ----------------------------
% {Matthias Fottner}                                                                      % Name des Protokollführers oder der Protokollführerin
% ----------------------------
{
  \begin{center}
    \begin{varwidth}{\textwidth}
    \begin{enumerate}
    \item Matthias Fottner
    \item David Keller
    \item  Moritz Woltron
    \end{enumerate}
    \end{varwidth}
\end{center}
}
% ----------------------------
{Helena Grabner}                                                                     %  Laborleiter(in)
% {Übung 2}                                                               %  Betreuer(in)
% ----------------------------
{Graz}                                                                                  %  Ort der Protokollerstellung
{\today}                                                                %  Datum Protokollerstellung


\newcommand{\unit}[1]{\,\text{#1}}


\tableofcontents

\newpage


\allowdisplaybreaks

\section{Bestimmen des Anfangszustands von $u_C$} % David
\subsection{Schaltplan zur Schalterposition a}

\subsection{Erstellen der erweiterten KSV-Matrix}
\subsection{Bestimmen von $u_C$}
\section{Aufstellen der Differentialgleichung} % Matthias
\subsection{Schaltplan zur Schalterposition b}
\begin{circuitikz}
  \clip (-0.8,-1) rectangle (11.5, 4.8);
  \draw (0,0) to[C, v<={\color{blue}{$u_C$}}, i<={\color{red}{$i_C$}}] ++ (0,4)
  to[short, -o] ++ (2,0)
  to[R=$R_5$, v<={\color{blue}{$u_{R5}$}}, i<={\color{red}{$i_{R5}$}}, o-*] ++ (4,0)
  to[R=$R_7$, v={\color{blue}{$U_{R7}$}}, i={\color{red}{$I_{R7}$}}, *-] ++ (4,0)
  to[isource, i={\color{red}{$I_{S3}$}}] ++ (0,-4)
  to[short, -o] ++ (-4, 0)
  to[short, *-o] ++ (-4,0)
  to[short, o-] ++ (-2, 0)
  (6,4) to[R=$R_6$, v={\color{blue}{$u_{R6}$}}, i={\color{red}{$i_{R6}$}}] (6,0)
  (6,0) node[rground]{};

  \draw[european voltages, color=green!50!black] (6.5, 4) to[open, v^=$U_{n1}$] (7,0);
  \draw[FARROW, green!50!black] (10.2, 3.8) to[bend left=80, looseness=1.7] node[below right]{$U_{n3}$} (6.5, -0.3);
  % \draw[gray,step=0.25] (-6,-3) grid (16.5, 11);
  % \draw [brown] (current bounding box.south west) rectangle (current bounding box.north east);
\end{circuitikz}

\begin{circuitikz}
  \draw (0,0) to[C, v<={\color{blue}{$u_C$}}] (0,4)
              to[short, -o] ++ (2,0)
              to[R=$R_{Th}$, v<={\color{blue}{$u_{RTh}$}}, i<={\color{red}{$i_C$}}] ++ (4,0)
              to[vsource, v={\color{blue}{$U_{Th}$}}] ++ (0,-4)
              to[short, -o] ++ (-4, 0)
              to[short] ++ (-2,0);
\end{circuitikz}

\subsection{Erstellen der KSV-Matrix}
\begin{align*}
  \renewcommand{\arraystretch}{1.5}
  \begin{bNiceArray}{c:c}
    G_6 + G_7 & -G_7 \\
    \hdottedline
    -G_7 & G_7
  \end{bNiceArray}
                            \begin{Bmatrix}
                              U_{n1} \\
                              U_{n2} \\
                            \end{Bmatrix} =
  \begin{Bmatrix}
    0 \\
    -I_{S3}
  \end{Bmatrix}
\end{align*}

\subsection{Lösen der Differentialgleichung}
\subsubsection{(Laplace Lösung)}
\setlength{\jot}{12pt}
\begin{align*}
  u_C + u_{RTh} &= U_{Th} \\
  u_C + R_{Th} \cdot i_C &= U_{Th} \\
  u_C + R_{Th} \cdot C \cdot u_C^{\prime} &= U_{Th} \\
  u_C^{\prime} + u_C \left( \frac{1}{R_{Th} \cdot C}\right) &= \frac{U_{Th}}{R_{Th} \cdot C} \\
                &\vLaplace \\
  s \cdot u_C(s) - u_C (0) + \left( \frac{1}{R_{Th} \cdot C} \right) u_C(s) &= \frac{U_{Th}}{R_{Th} \cdot C} \cdot \frac{1}{s} \\
  u_C(s) \left(s + \frac{1}{R_{Th} \cdot C}\right) &= \frac{U_{Th}}{R_{Th} \cdot C} \cdot \frac{1}{s} + u_C(0) \\
  u_C(s) &= \frac{U_{Th}}{R_{Th} \cdot C} \cdot\frac{1}{s \left( s + \frac{1}{R_{Th} \cdot C} \right)} + \frac{u_C(0)}{s + \frac{1}{R_{Th} \cdot C}} \\ \\
  \frac{1}{s\left(s + \frac{1}{R_{Th} \cdot C} \right)} &= \frac{A}{s} + \frac{B}{s + \frac{1}{R_{Th}\cdot C}} \\
  A &= \frac{1}{s + \frac{1}{R_{Th} \cdot C}} \Big|_{s=0} = R_{Th} \cdot C \\
  B &= \frac{1}{s} \Big|_{s=-\frac{1}{R_{Th} \cdot C}} = - R_{Th} \cdot C \\ \\
  u_C(s) &= \frac{U_{Th}}{R_{Th} \cdot C} \left( \frac{R_{Th} \cdot C}{s} - \frac{R_{Th} \cdot C}{s + \frac{1}{R_{Th} \cdot C}}\right) + \frac{u_C(0)}{s + \frac{1}{R_{Th} \cdot C}} \\
                &= U_{Th} \cdot \frac{1}{s} + \left( u_C(0) - U_{Th} \right) \frac{1}{s + \frac{1}{R_{Th} \cdot C}} \\
                &\vlaplace \\
  u_C(t) &= U_{Th} + (u_C(0) - U_{Th}) \cdot e^{-\frac{t}{R_{Th} \cdot C}} \\
\end{align*}

Die Funktion ist um $T_0$ nach rechts verschoben. Deswegen gilt:
\begin{align*}
  u_C(t) = \sigma(t-T_0) \left[ U_{Th} + (u_C(0) - U_{Th}) \cdot e^{-\frac{t-T_0}{R_{Th} \cdot C}} \right]
\end{align*}
\subsubsection{Homogene Lösung}

\subsubsection{Inhomogene Lösung}
\subsubsection{Anfangswertproblem}
\subsubsection{Gesamtlösung}
\section{Vergleich mit allgemeiner Lösungsformel}
\section{Simulation in PSpice}
\section{Matlab-Skript}


\end{document}