% Created 2020-10-25 So 23:03
% Intended LaTeX compiler: pdflatex
\documentclass[11pt]{scrartcl}
\usepackage[utf8]{inputenc}
\usepackage[T1]{fontenc}
\usepackage[ngerman]{babel}
\usepackage{graphicx}
\usepackage{grffile}
\usepackage{longtable}
\usepackage{wrapfig}
\usepackage{rotating}
\usepackage[normalem]{ulem}
\usepackage{amsmath}
\usepackage{textcomp}
\usepackage{amssymb}
\usepackage{capt-of}
\usepackage{MnSymbol}
\usepackage{mathtools}
\usepackage{setspace}
\usepackage{nicematrix}
\usepackage{varwidth}
% \usepackage{getlab}
\usepackage{listings}
\usepackage{pdfpages}
\usepackage[straightvoltages, european resistors, american inductors]{circuitikz}


\definecolor{darkspringgreen}{rgb}{0.09, 0.45, 0.27}    % Farbe für die Kommentare bei Listings
\lstset{
  language= Matlab,                     % Setzt die Sprache
  basicstyle=\scriptsize\ttfamily,     % Setzt den Standardstil
  % keywordstyle=\color{red}\bfseries,    % Setzt den Stil für Schlüsselwörter
  identifierstyle=\color{blue},        % Identifier bekommen keine gesonderte formatierung
  commentstyle=\color{darkspringgreen},        % Stil für Kommentare
  stringstyle=\ttfamily,             % Stil für Strings (gekennzeichnet mit "String")
  breaklines=true,             % Zeilen werden umgebrochen
  numbers=left,                 % Zeilennummern links
  numberstyle=\tiny,             % Stil für die Seitennummern
  frame=single,                 % Rahmen
  % backgroundcolor=\color{myGrey},     % Hintergrundfarbe
  % caption={Java-Code},             % Caption
  tabsize=2                % Größe der Tabulatoren
}



\newcommand{\GETHeader}[6]{%
  \begin{titlepage}
    \pagestyle{empty} \enlargethispage*{25cm}\samepage{
      \vspace*{-2.5cm}
      \begin{center}
        %
        \begin{tabular}[b]{lr}
          %
          \hspace*{-1.5cm}
          \begin{tabular}{p{3.8cm}}
            \includegraphics[width=3.8cm]{./Figures/TUGraz}
          \end{tabular}
          %
          \hspace*{-1cm}
          %
          %
          \begin{tabular}{p{13.8cm}}
            \begin{flushright}
              \large
              Institut für Grundlagen und Theorie der Elektrotechnik\\
              ~\\
            \end{flushright}
          \end{tabular}
        \end{tabular}

        \vspace*{2.2cm}
        %
        \Huge {Elektrische Netzwerke und Mehrtore \\ Übung\\} %
        \vspace*{.5cm} \Large{ Wintersemester 2020\\}
        %
        \vspace*{1.5cm}

        \Huge{\textbf{#1}\\}

        %
        % \vspace*{0.8cm} \Large{Übungsdatum: {#2}\\}
        %
        \vspace*{1cm} \vfill
        %
        \Large{Gruppe: {#2}\\} \vspace*{0.5cm}%


        % \Large{Protokollführer(in): {#3}\\} \vspace*{1cm}

        \Large{Gruppenteilnehmer:\\} \vspace*{.1cm}
        % Name der beteiligten Studierenden
        \Large{#3} \vspace{1cm}
        %
        %
        \Large{Vortragende: #4\\} \vspace*{.1cm}   %\vspace*{1.5cm}
        % \Large{Betreuer(in): #6\\}
        \vspace*{1.5cm}
        %
        \Large{#5, am #6}
      \end{center}}%
    %
    \clearpage
  \end{titlepage}}





\begin{document}

\GETHeader                                                                              %  Bitte Ausfüllen!!!
% ----------------------------
{Protokoll Übung 3: \\ Schaltvorgang Kondensator}                         %  Übungstitel
% ----------------------------
% {25. Mai 2020}                                                                  %  Übungsdatum
% ----------------------------
{04}                                                                   %  Gruppen-Nr.
% ----------------------------
% {Matthias Fottner}                                                                      % Name des Protokollführers oder der Protokollführerin
% ----------------------------
{
  \begin{center}
    \begin{varwidth}{\textwidth}
    \begin{enumerate}
    \item Matthias Fottner
    \item David Keller
    \item  Moritz Woltron
    \end{enumerate}
    \end{varwidth}
\end{center}
}
% ----------------------------
{Helena Grabner}                                                                     %  Laborleiter(in)
% {Übung 2}                                                               %  Betreuer(in)
% ----------------------------
{Graz}                                                                                  %  Ort der Protokollerstellung
{\today}                                                                %  Datum Protokollerstellung


\newcommand{\unit}[1]{\,\text{#1}}


\tableofcontents

\newpage



\section{Bestimmen des Anfangszustands von $u_C$}
\subsection{Schaltplan zur Schalterposition a}
\begin{figure}[!htb]
	\centering
	\hspace*{-40pt}
	\begin{circuitikz}[scale=0.85]
		\clip (-4.5,-2.5) rectangle (16.5, 11);
		% create all nodes
		\draw node[label=left:$n_1$](n1) at (0,4)
		node[label=south east:$n_2$](n2) at (4,4)
		node[label=south east:$n_3$](n3) at (8,4)
		node[label=north:$n_4$](n4) at (4,8)
		node[label=left:$n_5$](n5) at (0,8)
		node[label=north:$a$](a) at (10,8)
		node[label=south:$b$](b) at (10,4);
		
		% draw all resistors
		\draw (n5) to[R=$R_1$, v={\color{blue}{$U_{R1}$}}, i={\color{red}{$I_{R1}$}}, o-o] (n1)
		(n1) to[short] ++ (0.4, 0) to[R=$R_2$, v={\color{blue}{$U_{R2}$}}, i={\color{red}{$I_{R2}$}}] ++ (2.7, 0 ) to[short] (n2)
		(n2) to[R=$R_3$, v={\color{blue}{$U_{R3}$}}, i={\color{red}{$I_{R3}$}}, o-o] (n4)
		(n2) to[short] ++ (0,-1) to[R=$R_4$, v={\color{blue}{$U_{R4}$}}, i={\color{red}{$I_{R4}$}}] ++ (0,-2.5) to[short] (4,0)
		(8,8) to[R=$R_5$, v={\color{blue}{$U_{R5}$}}, i={\color{red}{$I_{R5}$}}, o-o] (n3)
		(4,0) to[R=$R_6$, v={\color{blue}{$U_{R6}$}}, i={\color{red}{$I_{R6}$}}, o-] (8,0)
		(12,8) to[R=$R_7$, v={\color{blue}{$U_{R7}$}}, i={\color{red}{$I_{R7}$}}, o-o] (12,4);
		
		% draw sources
		\draw (0,0) to[cvsource={\color{blue}{$U_{S1}$}}, i>={\color{red}{$I_{S1}^?$}}] (n1)
		(n4) to[vsource, v_={\color{blue}{$U_{S2}$}}, i>_={\color{red}{$I_{S2}^?$}}] (n5)
		(n2) to[short, o-] ++ (0.5,0) to[isource, v>={\color{blue}{$U_{S3}^?$}}, i>={\color{red}{$I_{S3}$}}] ++ (3,0) to[short, -o] (n3)
		(15,4) to[isource, v>={\color{blue}{$U_{S4}^?$}}, i={\color{red}{$I_{S4}$}}] (15,8);
		
		% fill in wires
		\draw (0,0) to[short] (4,0);
		\draw (8,0) to[short] (8,4);
		\draw (n4) to[short] ++ (4,0);
		\draw (n3) to[short, -o] ++ (2,0) to[short] ++ (2,0) to[short] ++ (3,0);
		\draw (8,8) to[short, -o] ++ (2,0) to[short] ++ (2,0) to[short] ++ (3,0);
		
		% draw ground
		\draw (4,0) node[rground]{};
		
		% draw node currents
		\draw[european voltages, color=green!50!black] (n2) to[open, v=$U_{n2}$] (2.1,0)
		(n1) to[open, v^=$U_{n1}$] (0.7,0)
		(-1.5,8) to[open, v=$U_{n5}$] (-1.5,0);
		
		
		% \draw[FARROW, green!50!black] plot [smooth, tension=1] coordinates {(8.7,3.3) (9,-1) (4,-1)}
		%\draw[FARROW, green!50!black] (8.6, 3.2) to[bend left=90, looseness=2] node[below right]{$U_{n3}$} (4.5,-1);
		%\draw[FARROW, green!50!black] (3, 9) to[bend right=110, looseness=1.9] node[below left]{$U_{n4}$} (0,-0.5);
		
		% \draw [brown] (current bounding box.south west) rectangle (current bounding box.north east);
		% \draw[gray,step=0.25] (-6,-3) grid (16.5, 11);
		
	\end{circuitikz}
	\caption{Netzwerk mit allen eingezeichneten Strömen, (Knoten-)spannungen und Knoten}
	\label{fig:net}
\end{figure}
\subsection{Erstellen der erweiterten KSV-Matrix}
\subsection{Bestimmen von $u_C$}
\section{Aufstellen der Differentialgleichung}
\subsection{Schaltplan zur Schalterposition b}
\subsection{Erstellen der KSV-Matrix}
\subsection{Lösen der Differentialgleichung}
\subsubsection{Homogene Lösung}
\subsubsection{Inhomogene Lösung}
\subsubsection{Anfangswertproblem}
\subsubsection{Gesamtlösung}
\section{Vergleich mit allgemeiner Lösungsformel}
\section{Simulation in PSpice}
\section{Matlab-Skript}


\end{document}