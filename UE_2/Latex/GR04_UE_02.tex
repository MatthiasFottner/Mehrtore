% Created 2020-10-25 So 23:03
% Intended LaTeX compiler: pdflatex
\documentclass[11pt]{scrartcl}
\usepackage[ngerman]{babel}
\usepackage{caption}
\usepackage[utf8]{inputenc}
\usepackage[T1]{fontenc}
\usepackage{graphicx}
\usepackage{grffile}
\usepackage{longtable}
\usepackage{wrapfig}
\usepackage{rotating}
\usepackage[normalem]{ulem}
\usepackage{amsmath}
\usepackage{textcomp}
\usepackage{amssymb}
\usepackage{capt-of}
\usepackage{MnSymbol}
\usepackage{mathtools}
\usepackage{setspace}
\usepackage{nicematrix}
\usepackage{listings}
\usepackage{pdfpages}
\usepackage[straightvoltages, european resistors, american inductors]{circuitikz}
\usetikzlibrary{arrows.meta}
\tikzset{FARROW/.style={arrows={-Triangle[angle=45:2.0mm]}}}


\renewcaptionname{ngerman}{\figurename}{Abb.}
\renewcaptionname{ngerman}{\tablename}{Tab.}



\author{David Keller, Moritz Woltron, Matthias Fottner}
\date{}
\title{Protokoll Übung 2}




\definecolor{darkspringgreen}{rgb}{0.09, 0.45, 0.27}    % Farbe für die Kommentare bei Listings
\lstset{
  language= Matlab,                     % Setzt die Sprache
  basicstyle=\scriptsize\ttfamily,     % Setzt den Standardstil
  % keywordstyle=\color{red}\bfseries,    % Setzt den Stil für Schlüsselwörter
  identifierstyle=\color{blue},        % Identifier bekommen keine gesonderte formatierung
  commentstyle=\color{darkspringgreen},        % Stil für Kommentare
  stringstyle=\ttfamily,             % Stil für Strings (gekennzeichnet mit "String")
  breaklines=true,             % Zeilen werden umgebrochen
  numbers=left,                 % Zeilennummern links
  numberstyle=\tiny,             % Stil für die Seitennummern
  frame=single,                 % Rahmen
  % backgroundcolor=\color{myGrey},     % Hintergrundfarbe
  % caption={Java-Code},             % Caption
  tabsize=2                % Größe der Tabulatoren
}




\begin{document}

\maketitle
\newcommand{\unit}[1]{\,\text{#1}}

\section{Schaltplan mit allen Strömen, Spannungen und Knoten}
\begin{figure}[!htb]
  \centering
\hspace*{-40pt}
\begin{circuitikz}[scale=0.85]
  \clip (-4.5,-2.5) rectangle (16.5, 11);
  % create all nodes
  \draw node[label=left:$n_1$](n1) at (0,4)
        node[label=south east:$n_2$](n2) at (4,4)
        node[label=south east:$n_3$](n3) at (8,4)
        node[label=north:$n_4$](n4) at (4,8)
        node[label=left:$n_5$](n5) at (0,8)
        node[label=north:$a$](a) at (10,8)
        node[label=south:$b$](b) at (10,4);

  % draw all resistors
  \draw (n5) to[R=$R_1$, v={\color{blue}{$U_{R1}$}}, i={\color{red}{$I_{R1}$}}, o-o] (n1)
        (n1) to[short] ++ (0.4, 0) to[R=$R_2$, v={\color{blue}{$U_{R2}$}}, i={\color{red}{$I_{R2}$}}] ++ (2.7, 0 ) to[short] (n2)
        (n2) to[R=$R_3$, v={\color{blue}{$U_{R3}$}}, i={\color{red}{$I_{R3}$}}, o-o] (n4)
        (n2) to[short] ++ (0,-1) to[R=$R_4$, v={\color{blue}{$U_{R4}$}}, i={\color{red}{$I_{R4}$}}] ++ (0,-2.5) to[short] (4,0)
        (8,8) to[R=$R_5$, v={\color{blue}{$U_{R5}$}}, i={\color{red}{$I_{R5}$}}, o-o] (n3)
        (4,0) to[R=$R_6$, v={\color{blue}{$U_{R6}$}}, i={\color{red}{$I_{R6}$}}, o-] (8,0)
        (12,8) to[R=$R_7$, v={\color{blue}{$U_{R7}$}}, i={\color{red}{$I_{R7}$}}, o-o] (12,4);

  % draw sources
  \draw (0,0) to[cvsource={\color{blue}{$U_{S1}$}}, i>={\color{red}{$I_{S1}^?$}}] (n1)
        (n4) to[vsource, v_={\color{blue}{$U_{S2}$}}, i>_={\color{red}{$I_{S2}^?$}}] (n5)
        (n2) to[short, o-] ++ (0.5,0) to[isource, v>={\color{blue}{$U_{S3}^?$}}, i>={\color{red}{$I_{S3}$}}] ++ (3,0) to[short, -o] (n3)
        (15,4) to[isource, v>={\color{blue}{$U_{S4}^?$}}, i={\color{red}{$I_{S4}$}}] (15,8);

  % fill in wires
  \draw (0,0) to[short] (4,0);
  \draw (8,0) to[short] (8,4);
  \draw (n4) to[short] ++ (4,0);
  \draw (n3) to[short, -o] ++ (2,0) to[short] ++ (2,0) to[short] ++ (3,0);
  \draw (8,8) to[short, -o] ++ (2,0) to[short] ++ (2,0) to[short] ++ (3,0);

  % draw ground
  \draw (4,0) node[rground]{};

  % draw node currents
  \draw[european voltages, color=green!50!black] (n2) to[open, v=$U_{n2}$] (2.1,0)
                                                 (n1) to[open, v^=$U_{n1}$] (0.7,0)
                                                 (-1.5,8) to[open, v=$U_{n5}$] (-1.5,0);


  % \draw[FARROW, green!50!black] plot [smooth, tension=1] coordinates {(8.7,3.3) (9,-1) (4,-1)}
  \draw[FARROW, green!50!black] (8.6, 3.2) to[bend left=90, looseness=2] node[below right]{$U_{n3}$} (4.5,-1);
  \draw[FARROW, green!50!black] (3, 9) to[bend right=110, looseness=1.9] node[below left]{$U_{n4}$} (0,-0.5);

  % \draw [brown] (current bounding box.south west) rectangle (current bounding box.north east);
  % \draw[gray,step=0.25] (-6,-3) grid (16.5, 11);

\end{circuitikz}
\caption{Netzwerk mit allen eingezeichneten Strömen, (Knoten-)spannungen und Knoten}
\label{fig:net}
\end{figure}

\section{Aufstellen des Gleichungssystems in Matrixform durch ``hinschauen''}
\subsection{Widerstände}
Zuerst wird die Admittanzmatrix durch hinschauen bestimmt.
Dazu werden auf der Hauptdiagonale alle den jeweiligen Knoten umgebenden Leitwerte addiert.
Auf der Nebendiagonale werden die Werte der zwischen den Knoten liegenden Widerstände als negativer Leitwert eingetragen.
Der Unbekanntenvektor besteht aus den Knotenspannungen, der Ergebnisvektor beträgt 0.

\NiceMatrixOptions{code-for-first-row = \color{orange},
                   code-for-first-col = \color{orange}}
\begin{align*}
  \renewcommand{\arraystretch}{2}
  \begin{bNiceArray}[first-row, first-col]{c:c:c:c:c}
          & n1 & n2 & n3 & n4 & n5 \\
          n1 & G_1+G_2 & -G_2 & 0 & 0 & -G_1 \\
          \hdottedline
          n2 & -G_2 & G_2 + G_3 + G_4 & 0 & -G_3 & 0 \\
          \hdottedline
          n3 & 0 & 0 & G_5 + G_6 & -G_5 & 0 \\
          \hdottedline
          n4 & 0 & -G_3 & -G_5 & G_3+G_5 & 0 \\
          \hdottedline
          n5 & -G_1 & 0 & 0 & 0 & G_1
        \end{bNiceArray}
   \begin{Bmatrix}
     U_{n1} \\
     U_{n2} \\
     U_{n3} \\
     U_{n4} \\
     U_{n5}
   \end{Bmatrix} =
  \begin{Bmatrix}
    0 \\
    0 \\
    0 \\
    0 \\
    0
  \end{Bmatrix}
\end{align*}
\subsection{Unabhängige Stromquelle $S3$}
Der bekannte Strom $I_{S3}$ der Stromquelle $S3$ fließt aus Knoten $n2$ (positiv) in den Knoten $n3$ (negativ)(vgl. Abb. \ref{fig:s3}).
Bringt man nun die Größe auf die andere Seite des Gleichungssystems,
so muss $I_{S3}$ im Lösungsvektor negativ bei $n2$ und positiv bei $n3$ sein.

\begin{figure}[!htb]
  \centering
\begin{circuitikz}
  \draw node[label=left:$n_2$](n2) at (0,0)
        node[label=right:$n_3$](n2) at (4,0);
  \draw (0,0) to[isource, v>={\color{blue}{$U_{S3}$}}, i={\color{red}{$I_{S3}$}}, o-o] (4,0);
\end{circuitikz}
\caption{freigestellte Stromquelle $S3$}
\label{fig:s3}
\end{figure}
Es ergibt sich folgende Änderung im Gleichungssystem:

\begin{align*}
  \renewcommand{\arraystretch}{2}
  \begin{bNiceArray}[first-row, first-col]{c:c:c:c:c}
    & n1 & n2 & n3 & n4 & n5 \\
    n1 & G_1+G_2 & -G_2 & 0 & 0 & -G_1 \\
    \hdottedline
    n2 & -G_2 & G_2 + G_3 + G_4 & 0 & -G_3 & 0 \\
    \hdottedline
    n3 & 0 & 0 & G_5 + G_6 & -G_5 & 0 \\
    \hdottedline
    n4 & 0 & -G_3 & -G_5 & G_3+G_5 & 0 \\
    \hdottedline
    n5 & -G_1 & 0 & 0 & 0 & G_1
  \end{bNiceArray}
                            \begin{Bmatrix}
                              U_{n1} \\
                              U_{n2} \\
                              U_{n3} \\
                              U_{n4} \\
                              U_{n5}
                            \end{Bmatrix} =
  \begin{Bmatrix}
    0 \\
    \color{red}{-I_{S3}} \\
    \color{red}{I_{S3}}\\
    0 \\
    0
  \end{Bmatrix}
\end{align*}

\subsection{Unabhängige Spannungsquelle $S2$ \color{green!50!black}{($\rightarrow +1$ Gleichung)}}
Bei der unabhängigen Spannungsquelle kommt mit dem Strom $I_{S2}^?$ eine unbekannte Größe hinzu.
Folglich muss auch eine weitere Gleichung im Gleichungssystem ergänzt werden.
Diese erhält man, indem die bekannte Quellspannung $U_{S2}$ mithilfe der Knotenspannungen ausgedrückt wird (vgl. Abb. \ref{fig:s2}):
\begin{equation*}
  {\color{blue}{U_{S2}}} = U_{n4} - U_{n5}
\end{equation*}
Beim Hinzufügen von $I_{S2}^?$ in den Unbekanntenvektor, muss auch die Strombilanz der Knoten $n4$ (positiv) und $n5$ (negativ) angepasst werden.

\begin{figure}[!htb]
  \centering
\begin{circuitikz}
  \draw node[label=left:$n_5$](n5) at (0,0)
        node[label=right:$n_4$](n4) at (4,0);
  \draw (4,0) to[vsource, v={\color{blue}{$U_{S2}$}}, i>={\color{red}{$I_{S2}^?$}}, o-o] (0,0);
\end{circuitikz}
\caption{freigestellte Spannungsquelle $S2$}
\label{fig:s2}
\end{figure}


\begin{align*}
  \renewcommand{\arraystretch}{2}
  \begin{bNiceArray}[first-row, first-col]{c:c:c:c:c:c}
    & n1 & n2 & n3 & n4 & n5 & \color{red}{I_{S2}^?}\\
    n1 & G_1+G_2 & -G_2 & 0 & 0 & -G_1 & 0\\
    \hdottedline
    n2 & -G_2 & G_2 + G_3 + G_4 & 0 & -G_3 & 0 & 0\\
    \hdottedline
    n3 & 0 & 0 & G_5 + G_6 & -G_5 & 0 & 0 \\
    \hdottedline
    n4 & 0 & -G_3 & -G_5 & G_3+G_5 & 0 & \color{red}{1} \\
    \hdottedline
    n5 & -G_1 & 0 & 0 & 0 & G_1 & \color{red}{-1} \\
    \hdottedline
    \color{red}{I_{S2}^?}   & 0 & 0 & 0 & \color{blue}{1} & \color{blue}{-1} & 0
  \end{bNiceArray}
                            \begin{Bmatrix}
                              U_{n1} \\
                              U_{n2} \\
                              U_{n3} \\
                              U_{n4} \\
                              U_{n5} \\
                              \color{red}{I_{S2}^?}
                            \end{Bmatrix} =
  \begin{Bmatrix}
    0 \\
    \color{red}{-I_{S3}} \\
    \color{red}{I_{S3}}\\
    0 \\
    0 \\
    \color{blue}{U_{S2}}
  \end{Bmatrix}
\end{align*}


\subsection{Stromgesteuerte Spannungsquelle $S1$ \color{green!50!black}{($\rightarrow +1$ Gleichung)}}
Bei der stromgesteuerten Spannungsquelle $S1$ kommt die unbekannte Größe $I_{S1}^?$ hinzu.
Folglich muss auch hier das Gleichungssystem um eine Gleichung erweitert werden.
\begin{align*}
  U_{S1} &= \alpha \cdot I_{R3} \\
  I_{R3} &= G_3 (U_{n2} - U_{n4}) \\
  U_{S1} &= -U_{n1} \\
  \Longrightarrow -U_{n1} &= \alpha \cdot G_3 (U_{n2}- U_{n4}) \\
  0 &= \alpha \cdot G_3 (U_{n2} - U_{n4}) + U_{n1}
\end{align*}

\begin{figure}[!htb]
  \centering
  \begin{circuitikz}
    \draw node[label=left:$n_1$](n1) at (0,4)
          node[label=right:$n_2$](n2) at (6,0)
          node[label=right:$n_4$](n4) at (6,4);

          \draw (0,4) to[short, o-] ++ (2,0) to[cvsource, i={\color{red}{$I_{S1}^?$}}, v_<={\color{blue}{$U_{S1}$}}] (2,0) to[short, -o] (0,0)
          (2,0) node[rground]{}
          (6,0) to[short, o-] (4,0) to[R=$R_3$, i={\color{red}{$I_{R3}$}}] (4,4) to[short, -o] (6,4);
  \end{circuitikz}
  \caption{freigestellte stromgesteuerte Spannungsquelle $S1$ mit Steuerungsstrom $I_{R3}$}
  \label{fig:s1}
\end{figure}

\begin{align*}
  \renewcommand{\arraystretch}{2}
  \begin{bNiceArray}[first-row, first-col]{c:c:c:c:c:c:c}
    & n1 & n2 & n3 & n4 & n5 & \color{red}{I_{S2}^?} & \color{red}{I_{S1}^?}\\
    n1 & G_1+G_2 & -G_2 & 0 & 0 & -G_1 & 0 & -1\\
    \hdottedline
    n2 & -G_2 & G_2 + G_3 + G_4 & 0 & -G_3 & 0 & 0 & 0\\
    \hdottedline
    n3 & 0 & 0 & G_5 + G_6 & -G_5 & 0 & 0 & 0\\
    \hdottedline
    n4 & 0 & -G_3 & -G_5 & G_3+G_5 & 0 & \color{red}{1} & 0 \\
    \hdottedline
    n5 & -G_1 & 0 & 0 & 0 & G_1 & \color{red}{-1} & 0\\
    \hdottedline
    \color{red}{I_{S2}^?} & 0 & 0 & 0 & \color{blue}{1} & \color{blue}{-1} & 0 & 0 \\
    \hdottedline
    \color{red}{I_{S1}^?} & 1 & \alpha G_3 & 0 & -\alpha G_3 & 0 & 0 & 0
  \end{bNiceArray}
                                                                               \begin{Bmatrix}
                                                                                 U_{n1} \\
                                                                                 U_{n2} \\
                                                                                 U_{n3} \\
                                                                                 U_{n4} \\
                                                                                 U_{n5} \\
                                                                                 \color{red}{I_{S2}^?} \\
                                                                                 \color{red}{I_{S1}^?}
                                                                               \end{Bmatrix} =
  \begin{Bmatrix}
    0 \\
    \color{red}{-I_{S3}} \\
    \color{red}{I_{S3}}\\
    0 \\
    0 \\
    \color{blue}{U_{S2}} \\
    0
  \end{Bmatrix}
\end{align*}

\end{document}